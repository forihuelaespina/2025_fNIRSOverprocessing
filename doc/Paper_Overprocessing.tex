\documentclass[12pt]{spieman}  % 12pt font required by SPIE;
%\documentclass[a4paper,12pt]{spieman}  % use this instead for A4 paper
\usepackage{amsmath,amsfonts,amssymb}
\usepackage{graphicx}
\usepackage{setspace}
\usepackage{tocloft}
\usepackage{lineno}
\linenumbers

%\usepackage{amsthm}
%\newtheorem{definition}{Definition}
%\newtheorem{theorem}{Theorem}[section]
%\newtheorem{corollary}{Corollary}[theorem]
%\newtheorem{lemma}[theorem]{Lemma}

\usepackage{color, colortbl} %For colored text
\usepackage{xcolor} %For defining new colors e.g. for equations
\definecolor{black}{RGB}{0,0,0}
\definecolor{equationscolor}{RGB}{0,0,0}
\usepackage{tikz} %create graphics programatically
\usetikzlibrary{arrows,shapes}
\usepackage{pgfplots} %Complements tikz to create scientific graphs e.g. use of axis etc.
\usepackage{hyperref}
\usepackage{caption}
\usepackage{subcaption} %Replace subfig but requires importing package caption explicitly. Provides environment \subfigure
%NOTE: Subfig and subcaption are incompatible. Only one can be used at a time.
\usepackage[normalem]{ulem} %for \sout

\newcommand{\mediaDir}{media}

\usepackage[framemethod=TikZ]{mdframed} %Permits shading entire paragraphs e.g. as in my new environment for exercises.


\newsavebox{\selvestebox}
\newenvironment{alertblock}
  {\begin{lrbox}{\selvestebox}%
   \begin{minipage}{%\dimexpr\columnwidth-2\fboxsep\relax
   0.9\linewidth
   }
   \color{black}}
  {\end{minipage}\end{lrbox} %
   \begin{center}
   %\setlength{\fboxsep}{0pt}
   %\fbox{
   	\colorbox[HTML]{EEEEEE}{\usebox{\selvestebox}}
    %}
   \end{center}}
   	%Solution found in: https://tex.stackexchange.com/questions/29769/colored-box-in-new-environment
    	%For some reason it interferes with \everymath so the colouring of the maths affects the text here







\title{Overprocessing in functional Near-Infrared Spectroscopy (fNIRS)}

\author[a]{Felipe Orihuela-Espina}
\affil[a]{University of Birmingham, School of Computer Science, Edgbaston, Birmingham, United Kingdom}
%\affil[b]{Company Name, Street Address, City, Country}

\renewcommand{\cftdotsep}{\cftnodots}
\cftpagenumbersoff{figure}
\cftpagenumbersoff{table} 
\begin{document} 
\maketitle

\begin{abstract}
\textbf{Significance}: %Provide the rationale or motivation for the work (i.e., its broad impact).
Overprocesssing occurs when we exceed on reasonable processing to extract the information from observations. Overprocessing can severely affect interpretation of results, e.g. increasing false positives.\\
\textbf{Aim}: %Briefly describe the study, tools, or systems used.
This paper introduces the problem of overprocessing to the fNIRS community.\\
\textbf{Approach}: %Briefly describe the materials and methods used.
The theoretical underpinnings revealing the existence of the problem are given, and the problem is formally stated. Two major avenues to approach the problem are presented.\\
\textbf{Results}: %Provide a core summary of study numbers, analyses, discoveries, or data descriptions.
The transfer function is discussed as a plausible and non-trivial processing and analysis pipeline that from an arbitrary experimental observation $\color{equationscolor}\mathbf{x}_i$ lands us into the hypothesis $\color{equationscolor}\mathbf{x}_h$. The analysis of such transfer function and the analysis of the problem geometry are discussed as potential ways to constraint the problem.\\
\textbf{Conclusions}: %An interpretive statement that summarizes the approach and results of the work.
At present, the fNIRS community lacks criteria to alleviate the risk of overprocessing. This draft intends to raise awareness on this largely unknown issue. 
\end{abstract}

% Include a list of up to six keywords after the abstract
\keywords{fNIRS, overprocessing, data analysis, signal processing, transfer function}

% Include email contact information for corresponding author
{\noindent \footnotesize\textbf{*}Felipe Orihuela-Espina,  \linkable{f.orihuela-espina@bham.ac.uk} }

\begin{spacing}{2}   % use double spacing for rest of manuscript

\begin{center}
\huge \textcolor{red}{Coming soon!}
\end{center}


\end{spacing}
\end{document}